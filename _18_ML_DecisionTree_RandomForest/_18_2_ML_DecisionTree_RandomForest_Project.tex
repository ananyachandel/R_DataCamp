\PassOptionsToPackage{unicode=true}{hyperref} % options for packages loaded elsewhere
\PassOptionsToPackage{hyphens}{url}
%
\documentclass[]{article}
\usepackage{lmodern}
\usepackage{amssymb,amsmath}
\usepackage{ifxetex,ifluatex}
\usepackage{fixltx2e} % provides \textsubscript
\ifnum 0\ifxetex 1\fi\ifluatex 1\fi=0 % if pdftex
  \usepackage[T1]{fontenc}
  \usepackage[utf8]{inputenc}
  \usepackage{textcomp} % provides euro and other symbols
\else % if luatex or xelatex
  \usepackage{unicode-math}
  \defaultfontfeatures{Ligatures=TeX,Scale=MatchLowercase}
\fi
% use upquote if available, for straight quotes in verbatim environments
\IfFileExists{upquote.sty}{\usepackage{upquote}}{}
% use microtype if available
\IfFileExists{microtype.sty}{%
\usepackage[]{microtype}
\UseMicrotypeSet[protrusion]{basicmath} % disable protrusion for tt fonts
}{}
\IfFileExists{parskip.sty}{%
\usepackage{parskip}
}{% else
\setlength{\parindent}{0pt}
\setlength{\parskip}{6pt plus 2pt minus 1pt}
}
\usepackage{hyperref}
\hypersetup{
            pdftitle={Tree Methods Project},
            pdfborder={0 0 0},
            breaklinks=true}
\urlstyle{same}  % don't use monospace font for urls
\usepackage[margin=1in]{geometry}
\usepackage{color}
\usepackage{fancyvrb}
\newcommand{\VerbBar}{|}
\newcommand{\VERB}{\Verb[commandchars=\\\{\}]}
\DefineVerbatimEnvironment{Highlighting}{Verbatim}{commandchars=\\\{\}}
% Add ',fontsize=\small' for more characters per line
\usepackage{framed}
\definecolor{shadecolor}{RGB}{248,248,248}
\newenvironment{Shaded}{\begin{snugshade}}{\end{snugshade}}
\newcommand{\AlertTok}[1]{\textcolor[rgb]{0.94,0.16,0.16}{#1}}
\newcommand{\AnnotationTok}[1]{\textcolor[rgb]{0.56,0.35,0.01}{\textbf{\textit{#1}}}}
\newcommand{\AttributeTok}[1]{\textcolor[rgb]{0.77,0.63,0.00}{#1}}
\newcommand{\BaseNTok}[1]{\textcolor[rgb]{0.00,0.00,0.81}{#1}}
\newcommand{\BuiltInTok}[1]{#1}
\newcommand{\CharTok}[1]{\textcolor[rgb]{0.31,0.60,0.02}{#1}}
\newcommand{\CommentTok}[1]{\textcolor[rgb]{0.56,0.35,0.01}{\textit{#1}}}
\newcommand{\CommentVarTok}[1]{\textcolor[rgb]{0.56,0.35,0.01}{\textbf{\textit{#1}}}}
\newcommand{\ConstantTok}[1]{\textcolor[rgb]{0.00,0.00,0.00}{#1}}
\newcommand{\ControlFlowTok}[1]{\textcolor[rgb]{0.13,0.29,0.53}{\textbf{#1}}}
\newcommand{\DataTypeTok}[1]{\textcolor[rgb]{0.13,0.29,0.53}{#1}}
\newcommand{\DecValTok}[1]{\textcolor[rgb]{0.00,0.00,0.81}{#1}}
\newcommand{\DocumentationTok}[1]{\textcolor[rgb]{0.56,0.35,0.01}{\textbf{\textit{#1}}}}
\newcommand{\ErrorTok}[1]{\textcolor[rgb]{0.64,0.00,0.00}{\textbf{#1}}}
\newcommand{\ExtensionTok}[1]{#1}
\newcommand{\FloatTok}[1]{\textcolor[rgb]{0.00,0.00,0.81}{#1}}
\newcommand{\FunctionTok}[1]{\textcolor[rgb]{0.00,0.00,0.00}{#1}}
\newcommand{\ImportTok}[1]{#1}
\newcommand{\InformationTok}[1]{\textcolor[rgb]{0.56,0.35,0.01}{\textbf{\textit{#1}}}}
\newcommand{\KeywordTok}[1]{\textcolor[rgb]{0.13,0.29,0.53}{\textbf{#1}}}
\newcommand{\NormalTok}[1]{#1}
\newcommand{\OperatorTok}[1]{\textcolor[rgb]{0.81,0.36,0.00}{\textbf{#1}}}
\newcommand{\OtherTok}[1]{\textcolor[rgb]{0.56,0.35,0.01}{#1}}
\newcommand{\PreprocessorTok}[1]{\textcolor[rgb]{0.56,0.35,0.01}{\textit{#1}}}
\newcommand{\RegionMarkerTok}[1]{#1}
\newcommand{\SpecialCharTok}[1]{\textcolor[rgb]{0.00,0.00,0.00}{#1}}
\newcommand{\SpecialStringTok}[1]{\textcolor[rgb]{0.31,0.60,0.02}{#1}}
\newcommand{\StringTok}[1]{\textcolor[rgb]{0.31,0.60,0.02}{#1}}
\newcommand{\VariableTok}[1]{\textcolor[rgb]{0.00,0.00,0.00}{#1}}
\newcommand{\VerbatimStringTok}[1]{\textcolor[rgb]{0.31,0.60,0.02}{#1}}
\newcommand{\WarningTok}[1]{\textcolor[rgb]{0.56,0.35,0.01}{\textbf{\textit{#1}}}}
\usepackage{graphicx,grffile}
\makeatletter
\def\maxwidth{\ifdim\Gin@nat@width>\linewidth\linewidth\else\Gin@nat@width\fi}
\def\maxheight{\ifdim\Gin@nat@height>\textheight\textheight\else\Gin@nat@height\fi}
\makeatother
% Scale images if necessary, so that they will not overflow the page
% margins by default, and it is still possible to overwrite the defaults
% using explicit options in \includegraphics[width, height, ...]{}
\setkeys{Gin}{width=\maxwidth,height=\maxheight,keepaspectratio}
\setlength{\emergencystretch}{3em}  % prevent overfull lines
\providecommand{\tightlist}{%
  \setlength{\itemsep}{0pt}\setlength{\parskip}{0pt}}
\setcounter{secnumdepth}{0}
% Redefines (sub)paragraphs to behave more like sections
\ifx\paragraph\undefined\else
\let\oldparagraph\paragraph
\renewcommand{\paragraph}[1]{\oldparagraph{#1}\mbox{}}
\fi
\ifx\subparagraph\undefined\else
\let\oldsubparagraph\subparagraph
\renewcommand{\subparagraph}[1]{\oldsubparagraph{#1}\mbox{}}
\fi

% set default figure placement to htbp
\makeatletter
\def\fps@figure{htbp}
\makeatother


\title{Tree Methods Project}
\author{}
\date{\vspace{-2.5em}}

\begin{document}
\maketitle

\hypertarget{tree-methods-project}{%
\section{Tree Methods Project}\label{tree-methods-project}}

For this project we will be exploring the use of tree methods to
classify schools as Private or Public based off their features. Let's
start by getting the data which is included in the ISLR library, the
College data frame. A data frame with 777 observations on the following
18 variables.

\begin{itemize}
\tightlist
\item
  Private A factor with levels No and Yes indicating private or public
  university
\item
  Apps Number of applications received
\item
  Accept Number of applications accepted
\item
  Enroll Number of new students enrolled
\item
  Top10perc Pct. new students from top 10\% of H.S. class
\item
  Top25perc Pct. new students from top 25\% of H.S. class
\item
  F.Undergrad Number of fulltime undergraduates
\item
  P.Undergrad Number of parttime undergraduates
\item
  Outstate Out-of-state tuition
\item
  Room.Board Room and board costs
\item
  Books Estimated book costs
\item
  Personal Estimated personal spending
\item
  PhD Pct. of faculty with Ph.D.'s
\item
  Terminal Pct. of faculty with terminal degree
\item
  S.F.Ratio Student/faculty ratio
\item
  perc.alumni Pct. alumni who donate
\item
  Expend Instructional expenditure per student
\item
  Grad.Rate Graduation rate
\end{itemize}

\hypertarget{get-the-data}{%
\subsection{Get the Data}\label{get-the-data}}

\hypertarget{call-the-islr-library-and-check-the-head-of-college-a-built-in-data-frame-with-islr-use-data-to-check-this.-then-reassign-college-to-a-dataframe-called-df}{%
\subsubsection{Call the ISLR library and check the head of College (a
built-in data frame with ISLR, use data() to check this.) Then reassign
College to a dataframe called
df}\label{call-the-islr-library-and-check-the-head-of-college-a-built-in-data-frame-with-islr-use-data-to-check-this.-then-reassign-college-to-a-dataframe-called-df}}

\begin{Shaded}
\begin{Highlighting}[]
\KeywordTok{remove}\NormalTok{(}\DataTypeTok{list =} \KeywordTok{ls}\NormalTok{())}
\KeywordTok{library}\NormalTok{(ISLR)}
\KeywordTok{head}\NormalTok{(College)}
\end{Highlighting}
\end{Shaded}

\begin{verbatim}
##                              Private Apps Accept Enroll Top10perc Top25perc
## Abilene Christian University     Yes 1660   1232    721        23        52
## Adelphi University               Yes 2186   1924    512        16        29
## Adrian College                   Yes 1428   1097    336        22        50
## Agnes Scott College              Yes  417    349    137        60        89
## Alaska Pacific University        Yes  193    146     55        16        44
## Albertson College                Yes  587    479    158        38        62
##                              F.Undergrad P.Undergrad Outstate Room.Board Books
## Abilene Christian University        2885         537     7440       3300   450
## Adelphi University                  2683        1227    12280       6450   750
## Adrian College                      1036          99    11250       3750   400
## Agnes Scott College                  510          63    12960       5450   450
## Alaska Pacific University            249         869     7560       4120   800
## Albertson College                    678          41    13500       3335   500
##                              Personal PhD Terminal S.F.Ratio perc.alumni Expend
## Abilene Christian University     2200  70       78      18.1          12   7041
## Adelphi University               1500  29       30      12.2          16  10527
## Adrian College                   1165  53       66      12.9          30   8735
## Agnes Scott College               875  92       97       7.7          37  19016
## Alaska Pacific University        1500  76       72      11.9           2  10922
## Albertson College                 675  67       73       9.4          11   9727
##                              Grad.Rate
## Abilene Christian University        60
## Adelphi University                  56
## Adrian College                      54
## Agnes Scott College                 59
## Alaska Pacific University           15
## Albertson College                   55
\end{verbatim}

\begin{Shaded}
\begin{Highlighting}[]
\NormalTok{df <-}\StringTok{ }\KeywordTok{data.frame}\NormalTok{(College)}
\KeywordTok{head}\NormalTok{(df)}
\end{Highlighting}
\end{Shaded}

\begin{verbatim}
##                              Private Apps Accept Enroll Top10perc Top25perc
## Abilene Christian University     Yes 1660   1232    721        23        52
## Adelphi University               Yes 2186   1924    512        16        29
## Adrian College                   Yes 1428   1097    336        22        50
## Agnes Scott College              Yes  417    349    137        60        89
## Alaska Pacific University        Yes  193    146     55        16        44
## Albertson College                Yes  587    479    158        38        62
##                              F.Undergrad P.Undergrad Outstate Room.Board Books
## Abilene Christian University        2885         537     7440       3300   450
## Adelphi University                  2683        1227    12280       6450   750
## Adrian College                      1036          99    11250       3750   400
## Agnes Scott College                  510          63    12960       5450   450
## Alaska Pacific University            249         869     7560       4120   800
## Albertson College                    678          41    13500       3335   500
##                              Personal PhD Terminal S.F.Ratio perc.alumni Expend
## Abilene Christian University     2200  70       78      18.1          12   7041
## Adelphi University               1500  29       30      12.2          16  10527
## Adrian College                   1165  53       66      12.9          30   8735
## Agnes Scott College               875  92       97       7.7          37  19016
## Alaska Pacific University        1500  76       72      11.9           2  10922
## Albertson College                 675  67       73       9.4          11   9727
##                              Grad.Rate
## Abilene Christian University        60
## Adelphi University                  56
## Adrian College                      54
## Agnes Scott College                 59
## Alaska Pacific University           15
## Albertson College                   55
\end{verbatim}

\hypertarget{eda}{%
\subsection{EDA}\label{eda}}

Let's explore the data!

\hypertarget{create-a-scatterplot-of-grad.rate-versus-room.board-colored-by-the-private-column.}{%
\subsubsection{Create a scatterplot of Grad.Rate versus Room.Board,
colored by the Private
column.}\label{create-a-scatterplot-of-grad.rate-versus-room.board-colored-by-the-private-column.}}

\begin{Shaded}
\begin{Highlighting}[]
\KeywordTok{library}\NormalTok{(ggplot2)}
\KeywordTok{ggplot}\NormalTok{(}\DataTypeTok{data =}\NormalTok{ df, }\KeywordTok{aes}\NormalTok{(Room.Board, Grad.Rate)) }\OperatorTok{+}\StringTok{ }\KeywordTok{geom_point}\NormalTok{(}\KeywordTok{aes}\NormalTok{(}\DataTypeTok{color =}\NormalTok{ Private))}
\end{Highlighting}
\end{Shaded}

\includegraphics{_18_2_ML_DecisionTree_RandomForest_Project_files/figure-latex/unnamed-chunk-2-1.pdf}
\#\#\# Create a histogram of full time undergrad students, color by
Private.

\begin{Shaded}
\begin{Highlighting}[]
\KeywordTok{ggplot}\NormalTok{(}\DataTypeTok{data =}\NormalTok{ df, }\KeywordTok{aes}\NormalTok{(F.Undergrad)) }\OperatorTok{+}\StringTok{ }\KeywordTok{geom_histogram}\NormalTok{(}\KeywordTok{aes}\NormalTok{(}\DataTypeTok{fill =}\NormalTok{ Private),}\DataTypeTok{color =} \StringTok{'black'}\NormalTok{, }\DataTypeTok{bins =} \DecValTok{50}\NormalTok{)}
\end{Highlighting}
\end{Shaded}

\includegraphics{_18_2_ML_DecisionTree_RandomForest_Project_files/figure-latex/unnamed-chunk-3-1.pdf}
\#\#\# Create a histogram of Grad.Rate colored by Private. You should
see something odd here.

\begin{Shaded}
\begin{Highlighting}[]
\KeywordTok{ggplot}\NormalTok{(}\DataTypeTok{data =}\NormalTok{ df, }\KeywordTok{aes}\NormalTok{(Grad.Rate)) }\OperatorTok{+}\StringTok{ }\KeywordTok{geom_histogram}\NormalTok{(}\KeywordTok{aes}\NormalTok{(}\DataTypeTok{fill =}\NormalTok{ Private),}\DataTypeTok{color =} \StringTok{'black'}\NormalTok{, }\DataTypeTok{bins =} \DecValTok{50}\NormalTok{)}
\end{Highlighting}
\end{Shaded}

\includegraphics{_18_2_ML_DecisionTree_RandomForest_Project_files/figure-latex/unnamed-chunk-4-1.pdf}
\#\#\# What college had a Graduation Rate of above 100\% ?

\begin{Shaded}
\begin{Highlighting}[]
\KeywordTok{subset}\NormalTok{(df,Grad.Rate }\OperatorTok{>}\StringTok{ }\DecValTok{100}\NormalTok{)}
\end{Highlighting}
\end{Shaded}

\begin{verbatim}
##                   Private Apps Accept Enroll Top10perc Top25perc F.Undergrad
## Cazenovia College     Yes 3847   3433    527         9        35        1010
##                   P.Undergrad Outstate Room.Board Books Personal PhD Terminal
## Cazenovia College          12     9384       4840   600      500  22       47
##                   S.F.Ratio perc.alumni Expend Grad.Rate
## Cazenovia College      14.3          20   7697       118
\end{verbatim}

\hypertarget{change-that-colleges-grad-rate-to-100}\label{change-that-colleges-grad-rate-to-100}}

\begin{Shaded}
\begin{Highlighting}[]
\NormalTok{df[}\StringTok{'Cazenovia College'}\NormalTok{, }\StringTok{'Grad.Rate'}\NormalTok{] <-}\StringTok{ }\DecValTok{100}
\end{Highlighting}
\end{Shaded}

\hypertarget{train-test-split}{%
\subsection{Train Test Split}\label{train-test-split}}

\hypertarget{split-your-data-into-training-and-testing-sets-7030.-use-the-catools-library-to-do-this.}{%
\subsubsection{Split your data into training and testing sets 70/30. Use
the caTools library to do
this.}\label{split-your-data-into-training-and-testing-sets-7030.-use-the-catools-library-to-do-this.}}

\begin{Shaded}
\begin{Highlighting}[]
\KeywordTok{set.seed}\NormalTok{(}\DecValTok{101}\NormalTok{)}
\KeywordTok{library}\NormalTok{(caTools)}

\NormalTok{sample <-}\StringTok{ }\NormalTok{caTools}\OperatorTok{::}\KeywordTok{sample.split}\NormalTok{(df}\OperatorTok{$}\NormalTok{Private, }\DataTypeTok{SplitRatio =} \FloatTok{0.70}\NormalTok{)}
\NormalTok{train <-}\StringTok{ }\KeywordTok{subset}\NormalTok{(df, sample }\OperatorTok{==}\StringTok{ }\OtherTok{TRUE}\NormalTok{)}
\NormalTok{test <-}\StringTok{ }\KeywordTok{subset}\NormalTok{(df, sample }\OperatorTok{==}\StringTok{ }\OtherTok{FALSE}\NormalTok{)}
\end{Highlighting}
\end{Shaded}

\hypertarget{decision-tree}{%
\subsection{Decision Tree}\label{decision-tree}}

\hypertarget{use-the-rpart-library-to-build-a-decision-tree-to-predict-whether-or-not-a-school-is-private.-remember-to-only-build-your-tree-off-the-training-data.}{%
\subsubsection{Use the rpart library to build a decision tree to predict
whether or not a school is Private. Remember to only build your tree off
the training
data.}\label{use-the-rpart-library-to-build-a-decision-tree-to-predict-whether-or-not-a-school-is-private.-remember-to-only-build-your-tree-off-the-training-data.}}

\begin{Shaded}
\begin{Highlighting}[]
\KeywordTok{library}\NormalTok{(rpart)}
\NormalTok{tree.model <-}\StringTok{ }\KeywordTok{rpart}\NormalTok{(Private}\OperatorTok{~}\NormalTok{., train, }\DataTypeTok{method =} \StringTok{'class'}\NormalTok{)}
\end{Highlighting}
\end{Shaded}

\hypertarget{use-predict-to-predict-the-private-label-on-the-test-data.}{%
\subsubsection{Use predict() to predict the Private label on the test
data.}\label{use-predict-to-predict-the-private-label-on-the-test-data.}}

\begin{Shaded}
\begin{Highlighting}[]
\NormalTok{predict.private <-}\StringTok{ }\KeywordTok{predict}\NormalTok{(}\DataTypeTok{object =}\NormalTok{ tree.model, test)}
\end{Highlighting}
\end{Shaded}

\hypertarget{check-the-head-of-the-predicted-values.-you-should-notice-that-you-actually-have-two-columns-with-the-probabilities.}{%
\subsubsection{Check the Head of the predicted values. You should notice
that you actually have two columns with the
probabilities.}\label{check-the-head-of-the-predicted-values.-you-should-notice-that-you-actually-have-two-columns-with-the-probabilities.}}

\begin{Shaded}
\begin{Highlighting}[]
\KeywordTok{head}\NormalTok{(predict.private)}
\end{Highlighting}
\end{Shaded}

\begin{verbatim}
##                                                  No       Yes
## Adrian College                          0.003311258 0.9966887
## Alfred University                       0.003311258 0.9966887
## Allegheny College                       0.003311258 0.9966887
## Allentown Coll. of St. Francis de Sales 0.003311258 0.9966887
## Alma College                            0.003311258 0.9966887
## Amherst College                         0.003311258 0.9966887
\end{verbatim}

\hypertarget{turn-these-two-columns-into-one-column-to-match-the-original-yesno-label-for-a-private-column.}{%
\subsubsection{Turn these two columns into one column to match the
original Yes/No Label for a Private
column.}\label{turn-these-two-columns-into-one-column-to-match-the-original-yesno-label-for-a-private-column.}}

\begin{Shaded}
\begin{Highlighting}[]
\NormalTok{predict.private <-}\StringTok{ }\KeywordTok{as.data.frame}\NormalTok{(predict.private)}
\CommentTok{# Lots of ways to do this}
\NormalTok{joiner <-}\StringTok{ }\ControlFlowTok{function}\NormalTok{(x)\{}
    \ControlFlowTok{if}\NormalTok{ (x}\OperatorTok{>=}\FloatTok{0.5}\NormalTok{)\{}
        \KeywordTok{return}\NormalTok{(}\StringTok{'Yes'}\NormalTok{)}
\NormalTok{    \}}\ControlFlowTok{else}\NormalTok{\{}
        \KeywordTok{return}\NormalTok{(}\StringTok{"No"}\NormalTok{)}
\NormalTok{    \}}
\NormalTok{\}}
\NormalTok{predict.private}\OperatorTok{$}\NormalTok{Private <-}\StringTok{ }\KeywordTok{sapply}\NormalTok{(predict.private}\OperatorTok{$}\NormalTok{Yes,joiner)}
\KeywordTok{head}\NormalTok{(predict.private)}
\end{Highlighting}
\end{Shaded}

\begin{verbatim}
##                                                  No       Yes Private
## Adrian College                          0.003311258 0.9966887     Yes
## Alfred University                       0.003311258 0.9966887     Yes
## Allegheny College                       0.003311258 0.9966887     Yes
## Allentown Coll. of St. Francis de Sales 0.003311258 0.9966887     Yes
## Alma College                            0.003311258 0.9966887     Yes
## Amherst College                         0.003311258 0.9966887     Yes
\end{verbatim}

\hypertarget{now-use-table-to-create-a-confusion-matrix-of-your-tree-model.}{%
\subsubsection{Now use table() to create a confusion matrix of your tree
model.}\label{now-use-table-to-create-a-confusion-matrix-of-your-tree-model.}}

\begin{Shaded}
\begin{Highlighting}[]
\KeywordTok{table}\NormalTok{(predict.private}\OperatorTok{$}\NormalTok{Private, test}\OperatorTok{$}\NormalTok{Private)}
\end{Highlighting}
\end{Shaded}

\begin{verbatim}
##      
##        No Yes
##   No   57   9
##   Yes   7 160
\end{verbatim}

\hypertarget{use-the-rpart.plot-library-and-the-prp-function-to-plot-out-your-tree-model.}{%
\subsubsection{Use the rpart.plot library and the prp() function to plot
out your tree
model.}\label{use-the-rpart.plot-library-and-the-prp-function-to-plot-out-your-tree-model.}}

\begin{Shaded}
\begin{Highlighting}[]
\KeywordTok{library}\NormalTok{(rpart.plot)}
\KeywordTok{prp}\NormalTok{(tree.model)}
\end{Highlighting}
\end{Shaded}

\includegraphics{_18_2_ML_DecisionTree_RandomForest_Project_files/figure-latex/unnamed-chunk-13-1.pdf}

\begin{Shaded}
\begin{Highlighting}[]
\KeywordTok{rpart.plot}\NormalTok{(tree.model)}
\end{Highlighting}
\end{Shaded}

\includegraphics{_18_2_ML_DecisionTree_RandomForest_Project_files/figure-latex/unnamed-chunk-14-1.pdf}
\#\# Random Forest

Now let's build out a random forest model!

\hypertarget{call-the-randomforest-package-library}{%
\subsubsection{Call the randomForest package
library}\label{call-the-randomforest-package-library}}

\begin{Shaded}
\begin{Highlighting}[]
\KeywordTok{library}\NormalTok{(randomForest)}
\end{Highlighting}
\end{Shaded}

\begin{verbatim}
## randomForest 4.6-14
\end{verbatim}

\begin{verbatim}
## Type rfNews() to see new features/changes/bug fixes.
\end{verbatim}

\begin{verbatim}
## 
## Attaching package: 'randomForest'
\end{verbatim}

\begin{verbatim}
## The following object is masked from 'package:ggplot2':
## 
##     margin
\end{verbatim}

\hypertarget{now-use-randomforest-to-build-out-a-model-to-predict-private-class.-add-importancetrue-as-a-parameter-in-the-model.-use-helprandomforest-to-find-out-what-this-does.}{%
\subsubsection{Now use randomForest() to build out a model to predict
Private class. Add importance=TRUE as a parameter in the model. (Use
help(randomForest) to find out what this
does.}\label{now-use-randomforest-to-build-out-a-model-to-predict-private-class.-add-importancetrue-as-a-parameter-in-the-model.-use-helprandomforest-to-find-out-what-this-does.}}

\begin{Shaded}
\begin{Highlighting}[]
\NormalTok{rf.model <-}\StringTok{ }\KeywordTok{randomForest}\NormalTok{(Private }\OperatorTok{~}\StringTok{ }\NormalTok{. , }\DataTypeTok{data =}\NormalTok{ train,}\DataTypeTok{importance =} \OtherTok{TRUE}\NormalTok{)}
\end{Highlighting}
\end{Shaded}

\hypertarget{what-was-your-models-confusion-matrix-on-its-own-training-set-use-modelconfusion.}{%
\subsubsection{What was your model's confusion matrix on its own
training set? Use
model\$confusion.}\label{what-was-your-models-confusion-matrix-on-its-own-training-set-use-modelconfusion.}}

\begin{Shaded}
\begin{Highlighting}[]
\NormalTok{rf.model}\OperatorTok{$}\NormalTok{confusion}
\end{Highlighting}
\end{Shaded}

\begin{verbatim}
##      No Yes class.error
## No  125  23  0.15540541
## Yes  10 386  0.02525253
\end{verbatim}

\hypertarget{grab-the-feature-importance-with-modelimportance.-refer-to-the-reading-for-more-info-on-what-gini1-means.2}{%
\subsubsection{Grab the feature importance with model\$importance. Refer
to the reading for more info on what Gini{[}1{]}
means.{[}2{]}}\label{grab-the-feature-importance-with-modelimportance.-refer-to-the-reading-for-more-info-on-what-gini1-means.2}}

\begin{Shaded}
\begin{Highlighting}[]
\NormalTok{rf.model}\OperatorTok{$}\NormalTok{importance}
\end{Highlighting}
\end{Shaded}

\begin{verbatim}
##                       No          Yes MeanDecreaseAccuracy MeanDecreaseGini
## Apps        0.0299975940 1.581643e-02         0.0195732992         9.093474
## Accept      0.0265169723 1.359972e-02         0.0169830344        11.252450
## Enroll      0.0360668986 2.852824e-02         0.0305094654        22.810173
## Top10perc   0.0094902570 5.700969e-03         0.0067307985         5.535670
## Top25perc   0.0049996299 2.891095e-03         0.0035060616         4.568160
## F.Undergrad 0.1591184093 6.945790e-02         0.0937286273        38.842710
## P.Undergrad 0.0436610506 7.334280e-03         0.0170748529        16.934128
## Outstate    0.1456738704 6.461743e-02         0.0865178571        42.669650
## Room.Board  0.0147668601 1.439793e-02         0.0144179025        10.414502
## Books       0.0007389787 8.914367e-05         0.0002868557         2.203342
## Personal    0.0029767520 8.632755e-04         0.0014664250         3.710353
## PhD         0.0086149598 5.320894e-03         0.0061979645         4.624097
## Terminal    0.0044437470 6.488106e-03         0.0059953078         4.524290
## S.F.Ratio   0.0302010312 8.307646e-03         0.0142830244        16.017899
## perc.alumni 0.0166161199 3.455476e-03         0.0071053253         5.280970
## Expend      0.0235732631 1.270668e-02         0.0155385819         9.860188
## Grad.Rate   0.0149251559 5.575670e-03         0.0080482732         6.520158
\end{verbatim}

\hypertarget{predictions}{%
\subsection{Predictions}\label{predictions}}

\hypertarget{now-use-your-random-forest-model-to-predict-on-your-test-set}{%
\subsubsection{Now use your random forest model to predict on your test
set!}\label{now-use-your-random-forest-model-to-predict-on-your-test-set}}

\begin{Shaded}
\begin{Highlighting}[]
\NormalTok{p <-}\StringTok{ }\KeywordTok{predict}\NormalTok{(rf.model,test)}
\KeywordTok{table}\NormalTok{(p,test}\OperatorTok{$}\NormalTok{Private)}
\end{Highlighting}
\end{Shaded}

\begin{verbatim}
##      
## p      No Yes
##   No   57   6
##   Yes   7 163
\end{verbatim}

\hypertarget{it-should-have-performed-better-than-just-a-single-tree-how-much-better-depends-on-whether-you-are-emasuring-recall-precision-or-accuracy-as-the-most-important-measure-of-the-model.}{%
\paragraph{It should have performed better than just a single tree, how
much better depends on whether you are emasuring recall, precision, or
accuracy as the most important measure of the
model.}\label{it-should-have-performed-better-than-just-a-single-tree-how-much-better-depends-on-whether-you-are-emasuring-recall-precision-or-accuracy-as-the-most-important-measure-of-the-model.}}

\end{document}
